\section{Introdução}

% XXX "facility" pode ser facilidade, mas "facilities" acredito que
% tenha uma chance menor de ser
O PQA (Problema Quadrático de Atribuição/Alocação), ou QAP (
\textit{Quadratic Assignment Problem}) \cite{qap-origin} é um dos
mais conhecidos e difíceis problemas de otimização combinatória
\cite{Maniezzo98exactand}. Entre suas aplicações, \cite{QAPBOOK} cita:
problemas de localização de instalações, escalonamento, problemas de
fiação em eletrônica, computação paralela e distribuída \cite{22},
analise estatística de dados, projeto de painéis de controle e teclados,
química, arqueologia, balanceamento de rotores de turbinas,
manufatura de computadores.
Em \cite{hospital} é abordado um problema real onde 30
instalações (\textit{facilities}) eram associadas a 30 locais no
hospital universitário Klinikum
Regensburg na Alemanha, problema este que é conhecido hoje como {\it
Kra30a}, desde que foi incluído na QAPLIB.

Coisas a arrumar no parágrafo anterior: \\
XXX A QAPLIB fala de uma publicação de 1978 sobre os problemas do krarup, por
que tem esse artigo de 2000 aqui ? XXX
XXX Não tinha sido dito nada da QAPLIB até agora, ficou estranho
colocar o nome dela no final do paragráfo XXX

Solucionar um QAP consiste em encontrar a melhor alocação de $n$
instalações para $n$ localizações. Matematicamente o problema pode ser
formulado como o de minimizar a seguinte função:
 \[ C(\pi) = \sum_{i = 1}^{n} \sum_{j = 1}^{n} f_{ij} d_{\pi(i)
   \pi(j)} \, , \quad \pi \in \prod(n)  \]
dadas duas matrizes, $F$ e $D$, de ordem $n \times n$, onde $\prod(n)$ é o
conjunto de todas as permutações de $\{1, \ldots , n\}$.

Devido a sua dificuldade, trabalhos publicados têm historicamente
notificado a solução exata de instâncias do QAP somente para valores
pequenos de $n$, com este variando entre
10 \cite{qap-origin} e 30 \cite{stutzle04}.
Por esse motivo, diversos métodos heurísticos têm sido propostos
para o QAP de modo a rapidamente produzir resultados aceitáveis.

Neste trabalho são utilizadas duas metaheurísticas para atacar este
problema: \begin{inparaenum}[(1)] \item Otimização por Colônia de
Formigas, ou {\it Ant Colony Optimization} (ACO) \cite{ACO}, que é
uma heurística inspirada no comportamento das formigas;
\item Algoritmo Memético, ou \textit{Memetic Algorithm} (AM) \cite{moscato1},
que opera baseado em uma população e tenta imitar a evolução
cultural\end{inparaenum}.

Este artigo foi dividido em três seções (além desta introdução). Na
sequência é fornecido um embasamento teórico que visa esclarecer
termos mencionados ao longo do trabalho. Depois discute-se as
implementações realizadas e finalmente os resultados obtidos são
apresentados.

XXX No parágrafo anterior, falar sobre os resultados obtidos


\section{Fundamentação teórica}
\label{sec:fund}

% XXX Só movi da introdução pra cá.
A matriz $F$ é chamada de matriz de fluxos, onde $f_{ij}$ representa o
fluxo de material da instalação $i$ para a $j$. A matriz $D$ é chamada
de matriz de distâncias, com $d_{kl}$ representando a distância da
localização $k$ a $l$. Alguns autores trazem também uma matriz $C$,
que realiza associação de custos, mas costuma ser ignorada por não
trazer contribuição significante para a complexidade do  problema
\cite{QACO}.

XXX

Detalhar um pouco mais o QAP.

Falar do custo de utilizar métodos para solução exata e
por isso formas alternativas tem sido utilizadas.
Falar que utilizou metaheurísticas e juntar com o parágrafo seguinte

% XXX Peguei do meu outro trabalho esse primeiro parágrafo.
% Mudar, complementar, etc.
De acordo com \cite{metatheory}, metaheurísticas são uma
classe de métodos aproximativos projetados para atacar problemas
difíceis de otimização combinatória que não obtiveram sucesso por meio
de heurísticas clássicas. Pode-se dizer que metaheurísticas são
heurísticas que guiam heurísticas, de modo que sejam combinados
conceitos diversos para exploração do espaço de busca permitindo a
fuga de ótimos locais.

Falar que o trabalhou utilizou ACO e AM, que são apresentados nas
Subseções seguintes.

\subsection{ACO}

\subsection{Algoritmos Meméticos}

Algoritmos Meméticos \cite{moscato1} são metaheurísticas baseadas em população.



\section{Implementação}



\section{Resultados}

XXX

Para avaliar a implementação, 20 instâncias, exibidas na Tabela
\ref{qapinst}, da QAPLIB \cite{qaplib} foram selecionadas. A coluna
\textit{gap} descreve a distância entre a melhor solução encontrada e
um limite inferior estabelecido. XXX


\begin{table}[H]
  \caption{Instâncias da QAPLIB utilizadas\label{qapinst}}
  \centering
  \begin{tabular}{l r r}
    \toprule
    Instância & Melhor solução & \textit{Gap} (\%) \\
    \midrule
    Nug12 & 578 & 0 \\
    Rou15 & 354210 & 0 \\
    Nug17 & 1732 & 0 \\
    Chr18b & 1534 & 0 \\
    Had20 & 6922 & 0 \\
    Nug22 & 3596 & 0 \\
    Chr25a & 3796 & 0 \\
    Tai25a & 1167256 & 12,94 \\
    Bur26b & 5426670 & 1,69 \\
    Nug30 & 6124 & 0 \\
    Kra30a & 88900 & 0 \\
    Kra32  & 88700 & 0 \\
    Esc32h & 438 & 21,00 \\
    Ste36c & & \\
    Tho40 & & \\
    Tai50b & 458821517 & 91,23 \\
    Wil50 & 48816 & 3,52 \\
    Tai60a & & \\
    Tai60b & & \\
    Esc64a & 116 & 59,49 \\
    \bottomrule
  \end{tabular}
\end{table}
