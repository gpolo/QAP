\section{Introdução}

O PQA (Problema Quadrático de Atribução/Alocação), ou QAP (
\textit{Quadratic Assignment Problem})\cite{qap-origin} é um dos
mais conhecidos e difíceis problemas de otimização combinatória
\cite{Maniezzo98exactand}. Entre suas aplicações, \cite{QAPBOOK} cita:
problemas de localização de facilidades, escalonamento, problemas de
fiação em eletrônica, computação paralela e distribuída \cite{22},
analise estatística de dados, projeto de painéis de controle e teclados,
química, arqueologia, \textit{balancing turbine runners}, manufatura de
computadores. Em \cite{hospital} é abordado um problema real onde 30
facilidades eram associadas a 30 locais no hospital universitário Klinikum
Regensburg na Alemanha, problema este que é conhecido hoje como {\it
Kra30a}, desde que foi incluído na QAPLIB.

% o que é o problema e a dificuldade 
Solucionar um QAP consiste em encontrar a melhor alocação de $n$
instalações para $n$ localizações. Matematicamente o problema pode ser
formulado como o de minimizar a seguinte função:

 \[ C(\pi) = \sum_{i = 1}^{n} \sum_{j = 1}^{n} f_{ij} d_{\pi(i)
   \pi(j)} \, , \quad \pi \in \prod(n)  \]
dadas duas matrizes, $F$ e $D$, de ordem $n \times n$, onde $\prod(n)$ é o
conjunto de todas as permutações de $\{1, \ldots , n\}$.

A matriz $F$ é chamada de matriz de fluxos, onde $f_{ij}$ representa o fluxo de
material da instalação $i$ para a $j$. A matriz $D$ é chamada de
matriz de distâncias, com $d_{kl}$ representando a distância da
localização $k$ a $l$. Alguns autores trazem também uma matriz $C$,
que realiza associação de custos, mas costuma ser ignorada por não
trazer contribuição significante para a complexidade do  problema
\cite{QACO}.

Como dito anteriormente o QAP é um dos mais difíceis problemas de
otimização combinatória, dificuldades camufladas por uma descrição
simples. Com o {\it hardware} disponível atualmente, algoritmos de solução
exata tem dificuldades em solucionar instâncias maiores do que $25$ \cite{QACO} e por
esse motivo diversas heurísticas tem sido propostas para o QAP, que não
garantem a melhor solução, porém produzem resultados bons em um tempo
aceitável.

Este trabalho fará um estudo comparativo utilizando duas meta-heuristicas
como ferramenta para solucionar o QAP. As meta-heuristicas escolhidas
neste trabalho são: {\it Ant Colony Optimization} (ACO) \cite{ACO}, uma heurística
inspirada no no comportamento das formigas e {\it Algoritmos Meméticos}
\cite{moscato1}, uma heurística que tenta imitar a evolução cultural.

Este trabalho está dividido da seguinte maneira: Na Seção \ref{sec:fund}
xxxx

\section{Fundamentação teórica}
\label{sec:fund}

Detalhar um pouco mais o QAP.

Falar do custo de utilizar métodos para solução exata e
por isso formas alternativas tem sido utilizadas.
Falar que utilizou metaheurísticas e juntar com o parágrafo seguinte

% XXX Peguei do meu outro trabalho esse primeiro parágrafo.
% Mudar, complementar, etc.
De acordo com \cite{metatheory}, metaheurísticas são uma
classe de métodos aproximativos projetados para atacar problemas
difíceis de otimização combinatória que não obtiveram sucesso por meio
de heurísticas clássicas. Pode-se dizer que metaheurísticas são
heurísticas que guiam heurísticas, de modo que sejam combinados
conceitos diversos para exploração do espaço de busca permitindo a
fuga de ótimos locais.

Falar que o trabalhou utilizou ACO e AM, que são apresentados nas
Subseções seguintes.

\subsection{ACO}

\subsection{Algoritmos Meméticos}



\section{Implementação}



\section{Resultados}

XXX

Para avaliar a implementação, 20 instâncias, exibidas na Tabela
\ref{qapinst}, da QAPLIB \cite{qaplib} foram selecionadas. A coluna
\textit{gap} descreve a distância entre a melhor solução encontrada e
um limite inferior estabelecido. XXX

\begin{table}[H]
  \caption{Instâncias da QAPLIB utilizadas\label{qapinst}}
  \centering
  \begin{tabular}{l r r}
    \toprule
    Instância & Melhor solução & \textit{Gap} (\%) \\
    \midrule
    Nug12 & 578 & 0 \\
    Rou15 & 354210 & 0 \\
    Nug17 & 1732 & 0 \\
    Chr18b & 1534 & 0 \\
    Had20 & 6922 & 0 \\
    Nug22 & 3596 & 0 \\
    Chr25a & 3796 & 0 \\
    Tai25a & 1167256 & 12,94 \\
    Bur26b & 5426670 & 1,69 \\
    Nug30 & 6124 & 0 \\
    Kra30a & 88900 & 0 \\
    Kra32  & 88700 & 0 \\
    Esc32h & 438 & 21,00 \\
    Tai50b & 458821517 & 91,23 \\
    Wil50 & 48816 & 3,52 \\
    Esc64a & 116 & 59,49 \\
    \bottomrule
  \end{tabular}
\end{table}
