\section{Introdução}

O PQA (Problema Quadrático de Atribuição/Alocação), ou QAP (
\textit{Quadratic Assignment Problem}) \cite{qap-origin} é um dos
mais conhecidos e difíceis problemas de otimização combinatória
\cite{Maniezzo98exactand}. Entre suas aplicações, \cite{QAPBOOK} cita:
problemas de localização de instalações, escalonamento, problemas de
fiação em eletrônica, computação paralela e distribuída \cite{22},
analise estatística de dados, projeto de painéis de controle e teclados,
química, arqueologia, balanceamento de rotores de turbinas e
manufatura de computadores.
% Em \cite{hospital} é abordado um problema real onde 30
% instalações (\textit{facilities}) eram associadas a 30 locais no
% hospital universitário Klinikum
% Regensburg na Alemanha, problema este que é conhecido hoje como {\it
% Kra30a}, desde que foi incluído na QAPLIB.

%Coisas a arrumar no parágrafo anterior: \\
%XXX A QAPLIB fala de uma publicação de 1978 sobre os problemas do krarup, por
%que tem esse artigo de 2000 aqui ? XXX
%XXX Não tinha sido dito nada da QAPLIB até agora, ficou estranho
%colocar o nome dela no final do paragráfo XXX

Solucionar um QAP consiste em encontrar a melhor alocação de $n$
instalações para $n$ localizações. Matematicamente o problema pode ser
formulado como o de minimizar a seguinte função:
\begin{equation}\label{calc-custo}
C(\pi) = \sum_{i = 1}^{n} \sum_{j = 1}^{n} f_{ij} d_{\pi(i)
   \pi(j)} \, , \quad \pi \in \prod(n)
\end{equation}
dadas duas matrizes, $F$ e $D$, de ordem $n \times n$, onde $\prod(n)$ é o
conjunto de todas as permutações de $\{1, \ldots , n\}$.

Devido a sua dificuldade, trabalhos publicados têm historicamente
notificado a solução exata de instâncias do QAP somente para valores
pequenos de $n$, com este variando entre
10 \cite{qap-origin} e 30 \cite{stutzle04}.
Por esse motivo, diversos métodos heurísticos têm sido propostos
para o QAP de modo a rapidamente produzir resultados aceitáveis.

Neste trabalho, a abordagem ao problema ocorre com uso de duas
metaheurísticas: \begin{inparaenum}[(1)] \item Otimização por Colônia
  de Formigas, ou {\it Ant Colony Optimization} (ACO) \cite{ACO}, que é
uma heurística inspirada no comportamento das formigas;
\item Algoritmo Memético (AM), ou \textit{Memetic Algorithm}\cite{moscato1},
que opera baseado em uma população e tenta imitar a evolução
cultural\end{inparaenum}.

Este artigo foi dividido em três seções (além desta introdução). Na
sequência é fornecido um embasamento teórico que visa esclarecer
termos mencionados ao longo do trabalho. Depois discute-se as
implementações realizadas e finalmente os resultados obtidos são
apresentados.

XXX No parágrafo anterior, falar sobre os resultados obtidos


\section{Fundamentação teórica}
\label{sec:fund}

O QAP é um problema de otimização classificado como
$\mathcal{NP}$-difícil, 
% XXX Seria bom indicar qual trabalho provou isso (NP-dificil)
portanto desconhece-se um algoritmo de tempo polinomial que resolve
este problema. Aproximações, para qualquer instância, para este
problema também foram classificadas como $\mathcal{NP}$-difícil.
% XXX Essa última frase provavelmente precisa de uma explicação
% melhor.

...

% XXX Só movi da introdução pra cá.
A matriz $F$ é chamada de matriz de fluxos, onde $f_{ij}$ representa o
fluxo de material da instalação $i$ para a $j$. A matriz $D$ é chamada
de matriz de distâncias, com $d_{kl}$ representando a distância da
localização $k$ a $l$. Alguns autores trazem também uma matriz $C$,
que realiza associação de custos, mas costuma ser ignorada por não
trazer contribuição significante para a complexidade do  problema
\cite{QACO}.

XXX

Detalhar um pouco mais o QAP.

Falar de como o problema pode ser representado e também como uma
solução é representada neste trabalho.

Falar do custo de utilizar métodos para solução exata e
por isso formas alternativas tem sido utilizadas.
Falar que utilizou metaheurísticas e juntar com o parágrafo seguinte

% XXX Peguei do meu outro trabalho esse primeiro parágrafo.
% Mudar, complementar, etc.
De acordo com \cite{metatheory}, metaheurísticas são uma
classe de métodos aproximativos projetados para atacar problemas
difíceis de otimização combinatória que não obtiveram sucesso por meio
de heurísticas clássicas. Pode-se dizer que metaheurísticas são
heurísticas que guiam heurísticas, de modo que sejam combinados
conceitos diversos para exploração do espaço de busca permitindo a
fuga de ótimos locais.

Falar que o trabalhou utilizou ACO e AM, que são apresentados nas
Subseções seguintes.

\subsection{ACO}

\subsection{Algoritmos Meméticos}

Algoritmos Meméticos \cite{moscato1} são metaheurísticas baseadas
em população e evolução cultural.
Seu nome é derivado do termo \textit{meme} \cite{dawkins}, que é
definido por \cite{oxford-dict} como um elemento cutural que consegue
replicar-se por meios não-genéticos, como a imitação.
%an element of a culture or system of behaviour that may be considered
%to replicate by passing from one individual to another by nongenetic
%means, esp. imitation (página 271 do dicionário)
O trabalho de \cite{moscato1} faz uma comparação entre algoritmos
meméticos e artes marciais, tratando os movimentos coordenados e
eficazes dos mestres do Kung-Fu como algo que foi sendo aperfeiçoado
a partir do conhecimento formado por cada mestre. Este mesmo trabalho
também menciona que:
\begin{quote}
\textit{``A scientist does not pass on an idea after blending it with
  with his own without checking the logic of what he is saying or his
  reputation would be in trouble. Altough there there are some
  exceptions, science does not improve by random errors.''}
\end{quote}
Com estas analogias à AM deve ser possível observar que a mutação, ao
contrário de algoritmos genéticos (AG), não é um operador
característico desta metaheurística.

Neste método, uma população (um conjunto de soluções, ou cromossomos
em AG, factíveis) é
inicialmente criada. Cada indivíduo (uma solução factível) é um ótimo
local. A cada geração (iteração) trabalha-se com recombinações,
seguido de obtenção de ótimo local, de indivíduos da população
atual. O operador de mutação é utilizado no caso de uma
geração apresentar baixa diversidade, ou seja: os indivíduos estão
mais semelhantes que um valor limite estabelecido.
%geração apresentar baixa diversidade, ou seja: os alelos ..  genes
%(elementos que compõem uma solução) na população estão
%suficientemente parecidos. XXX muita notação à toa..
Ao estabelecer um limite para o tamanho do conjunto de soluções,
torna-se necessário o uso de uma função que seleciona os indivíduos
mais adaptados dos demais para permanecerem na população.


\section{Implementação}

\subsection{Puramente aleatória}

A obtenção de soluções de instâncias do QAP por meio de um algoritmo
puramente aleatório é simples. Apesar desta forma de resolução não
fazer parte na coleta de resultados, ela é útil na discussão a
respeito da representação de soluções e custos envolvidos.

\begin{codebox}
\Procname{$\proc{qap-random}(F, D)$}
\li $\id{melhor-sol} \gets \emptyset$
\li $\id{melhor-custo} \gets \infty$
\li \While critérios de parada não satisfeitos \Do \label{criterio-random}
\li     $\id{sol} \gets \proc{solução-aleatória}()$
\li	$\id{custo} \gets \proc{calc-custo}(F, D, \id{sol})$
\li	\If $\id{custo} < \id{melhor-custo}$ \Then
\li		$\id{melhor-custo} \gets \id{custo}$
\li		$\id{melhor-sol} \gets \id{sol}$
        \End
    \End
\li \Return ($\id{melhor-sol}, \id{melhor-custo}$)
\end{codebox}

A função \proc{qap-random} recebe as duas matrizes que descrevem uma
instância QAP e retorna a melhor solução, com seu respectivo custo,
encontrada. Se implementado, critérios de parada (linha
\ref{criterio-random}) podem ser quantidade de iterações ou quantidade
de iterações sem redução de custo. Uma solução é construída com uso da
função \proc{solução-aleatória}, sendo esta também utilizada em certos
momentos das metaheurísticas desenvolvidas aqui. Calcular o custo de
cada solução envolve o uso da Equação \ref{calc-custo}, cuja
complexidade computacional é $\Theta(n^2)$. Logo, para ser
compatível com esta equação, uma solução precisa ser uma das $n!$
permutações possíveis para uma instância com matrizes de ordem
$n \times n$.

Representar uma solução qualquer na forma de permutação, como por
exemplo: $[1\,\, 4\,\, 2\,\, 3]$, tem o significado de atribuir a
instalação 1 na localização 1, instalação 4 na localização 2 e assim
por diante. De forma geral, a instalação $\pi(i)$ fica
localizada em $i$. Esta é uma forma simples de representar soluções no
QAP, sendo necessário cuidados para evitar o custo da Equação
\ref{calc-custo} caso ela seja modificada em métodos de busca local,
mas foi utilizada ao longo deste trabalho.

\subsection{ACO}

\subsection{AM}



\section{Resultados}

XXX

Para avaliar a implementação, 20 instâncias, exibidas na Tabela
\ref{qapinst}, da QAPLIB \cite{qaplib} foram selecionadas. A coluna
\textit{gap} descreve a distância entre a melhor solução encontrada e
um limite inferior estabelecido. XXX


\begin{table}[H]
  \caption{Instâncias da QAPLIB utilizadas\label{qapinst}}
  \centering
  \begin{tabular}{l r r}
    \toprule
    Instância & Melhor solução & \textit{Gap} (\%) \\
    \midrule
    Nug12 & 578 & 0 \\
    Rou15 & 354210 & 0 \\
    Nug17 & 1732 & 0 \\
    Chr18b & 1534 & 0 \\
    Had20 & 6922 & 0 \\
    Nug22 & 3596 & 0 \\
    Chr25a & 3796 & 0 \\
    Tai25a & 1167256 & 12,94 \\
    Bur26b & 5426670 & 1,69 \\
    Nug30 & 6124 & 0 \\
    Kra30a & 88900 & 0 \\
    Kra32  & 88700 & 0 \\
    Esc32h & 438 & 21,00 \\
    Ste36c & & \\
    Tho40 & & \\
    Tai50b & 458821517 & 91,23 \\
    Wil50 & 48816 & 3,52 \\
    Tai60a & & \\
    Tai60b & & \\
    Esc64a & 116 & 59,49 \\
    \bottomrule
  \end{tabular}
\end{table}
