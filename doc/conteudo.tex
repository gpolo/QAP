\section{Introdução}

O PQA (Problema Quadrático de Atribuição/Alocação), ou QAP
(\textit{Quadratic Assignment Problem}) \cite{qap-origin}, consiste em
minimizar a função
 \[ C(\pi) = \sum_{i = 1}^{n} \sum_{j = 1}^{n} f_{ij} d_{\pi(i)
   \pi(j)} \, , \quad \pi \in \prod(n)  \]
dadas duas matrizes, $F$ e $D$, de ordem $n \times n$, onde $\prod(n)$ é o
conjunto de todas as permutações de $\{1, \ldots , n\}$.

Apesar da simples descrição e de resolução trivial (Apêndice A) fácil,
este problema é um dos mais conhecidos e difíceis da otimização
combinatória \cite{Maniezzo98exactand}. Sua resolução consiste em
encontrar a melhor alocação de $n$ instalações para $n$ localizações,
mesmo pequenas instâncias do problema têm sido difíceis de
resolverem. Matematicamente, o problema é definido pelas matrizes $F$,
$D$ e $C$, todas $n \times n$. Costuma-se encontrar a matriz $F$ como
sendo chamada de matriz de fluxos, onde $f_{ij}$ representa o fluxo de
material da instalação $i$ para a $j$. A matriz $D$ é chamada de
matriz de distâncias, com $d_{kl}$ representando a distância da
localização $k$ a $l$. Alguns autores trazem também uma matriz $C$,
que realiza associação de custos, mas costuma ser ignorada por não
trazer contribuição significante para a complexidade do  problema
\cite{QACO}.

Desta forma, propõe-se o uso de duas meta-heurísticas para tentar
encontrar soluções razoáveis para instâncias do QAP. A primeira
meta-heurística, a \textit{Ant Colony Optimization} (ACO) \cite{ACO},
é inspirada no comportamento das formigas. A ACO é projetada para ser
uma meta-heurística que guie a busca global por meio da atualização do
``conhecimento’’ adquirido pelo sistema durante o histórico de busca
\cite{QACO}.
A segunda proposta faz uso de Algoritmos Meméticos \cite{moscato1} com
busca local de heurística \textit{pairwise interchange}. Algoritmos
Meméticos tentam imitar a evolução cultural, formando uma união entre
a busca global baseada em população e a heurística de busca local
realizada por cada um dos indivíduos. Esta meta-heurística tem
alcançado resultados significativos em problemas de otimização
variados \cite{moscato2}, inclusive o QAP
\cite{merz_freisleben}. Assim, espera-se que seu uso possa trazer bons
resultados neste trabalho.


\section{Fundamentação teórica}

Detalhar um pouco mais o QAP.

Falar do custo de utilizar métodos para solução exata e
por isso formas alternativas tem sido utilizadas.
Falar que utilizou metaheurísticas e juntar com o parágrafo seguinte

% XXX Peguei do meu outro trabalho esse primeiro parágrafo.
% Mudar, complementar, etc.
De acordo com \cite{metatheory}, metaheurísticas são uma
classe de métodos aproximativos projetados para atacar problemas
difíceis de otimização combinatória que não obtiveram sucesso por meio
de heurísticas clássicas. Pode-se dizer que metaheurísticas são
heurísticas que guiam heurísticas, de modo que sejam combinados
conceitos diversos para exploração do espaço de busca permitindo a
fuga de ótimos locais.

Falar que o trabalhou utilizou ACO e AM, que são apresentados nas
Subseções seguintes.

\subsection{ACO}

\subsection{Algoritmos Meméticos}



\section{Implementação}



\section{Resultados}

XXX

Para avaliar a implementação, 20 instâncias, exibidas na Tabela
\ref{qapinst}, da QAPLIB \cite{qaplib} foram selecionadas. A coluna
\textit{gap} descreve a distância entre a melhor solução encontrada e
um limite inferior estabelecido. XXX

\begin{table}[H]
  \caption{Instâncias da QAPLIB utilizadas\label{qapinst}}
  \centering
  \begin{tabular}{l r r}
    \toprule
    Instância & Melhor solução & \textit{Gap} (\%) \\
    \midrule
    Nug12 & 578 & 0 \\
    Rou15 & 354210 & 0 \\
    Nug17 & 1732 & 0 \\
    Chr18b & 1534 & 0 \\
    Had20 & 6922 & 0 \\
    Nug22 & 3596 & 0 \\
    Chr25a & 3796 & 0 \\
    Tai25a & 1167256 & 12,94 \\
    Bur26b & 5426670 & 1,69 \\
    Nug30 & 6124 & 0 \\
    Kra30a & 88900 & 0 \\
    Kra32  & 88700 & 0 \\
    Esc32h & 438 & 21,00 \\
    Tai50b & 458821517 & 91,23 \\
    Wil50 & 48816 & 3,52 \\
    Esc64a & 116 & 59,49 \\
    \bottomrule
  \end{tabular}
\end{table}
